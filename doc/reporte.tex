\documentclass[40pt,a4paper]{article}
\usepackage[margin=0.7in]{geometry}
\usepackage[utf8]{inputenc}
\usepackage{graphicx}
\usepackage{hyperref}
\usepackage{courier}

%\usepackage[monochrome]{color}

\usepackage{listings}
\usepackage{color}

\definecolor{dkgreen}{rgb}{0,0.6,0}
\definecolor{gray}{rgb}{0.5,0.5,0.5}
\definecolor{mauve}{rgb}{0.58,0,0.82}
\definecolor{orange}{rgb}{0.8,0.3,0}

\lstset{frame=tb,
  language=Java,
  aboveskip=3mm,
  belowskip=3mm,
  showstringspaces=false,
  columns=flexible,
  basicstyle={\small\ttfamily},
  numbers=none,
  numberstyle=\tiny\color{gray},
  keywordstyle=\color{blue},
  commentstyle=\color{dkgreen},
  stringstyle=\color{mauve},
  breaklines=true,
  breakatwhitespace=true,
  tabsize=3,
  literate={μ}{{$\mu$ }}1 {²}{{\textsuperscript{2}}}1 {σ}{{$\sigma$}}1 {á}{{\'a}}1 {ñ}{{\~n}}1 {é}{{\'e}}1 {í}{{\'i}}1 {ó}{{\'o}}1 {ú}{{\'u}}1,
}

\lstdefinelanguage{JavaScript}{
  keywords={typeof, new, true, false, catch, function, return, null, catch, switch, var, if, in, while, do, else, case, break, try, document, getElementById, innerHTML},
  keywordstyle=\color{orange}\bfseries,
  ndkeywords={class, export, boolean, throw, implements, import, this},
  ndkeywordstyle=\color{gray}\bfseries,
  identifierstyle=\color{black},
  sensitive=false,
  comment=[l]{//},
  morecomment=[s]{/*}{*/},
  commentstyle=\color{red}\ttfamily,
  stringstyle=\color{red}\ttfamily,
  morestring=[b]',
  morestring=[b]"
}

\begin{document}

\title{
Análisis de Algoritmos\\
{\small Mario Augusto Ramirez Morales\\}
Proyecto\\
}
\author{
Grupo 3CV5\\ \\
Equipo\\ \\
Díaz Urbina Eduardo 2012630487\\
}
\date{4 de Diciembre 2013}
\maketitle

\section{Objetivo}

Obtener el recorrido entre dos palabras implementando los algoritmos
vistos en clase.

Se entendiende por recorrido entre dos palabras la lista ordenada
de palabras que cumplen:

\begin{itemize}
	\item Primer y último elemento son dichas dos palabras
	\item Dado cualquier elemento excepto el último, el elemento
	      siguiente difiere de él en una letra solamente
\end{itemize}

\section{Introducción}

Se va a trató cada palabra como un nodo y la lista de palabras se
obtuvieron de diccionarios de palabras.

Se implementó en JavaScript + HTML, y se recomienda usar la última
versión de Chrome o Firefox usando la consola presionando
Ctrl+Mayus+J y Ctr+Mayus+K respectivamente.

\paragraph{Algoritmos}
Solo se implementaron los algoritmos de búsqueda en profundidad y
búsqueda en amplitud.

\paragraph{Búsqueda en profundidad}
Es un algoritmo que recorre los nodos de un grafo de manera que
expande la búsqueda lo más posible dentro del grafo, generalmente
no encuentra rutas cortas y es implementado con ayuda de una estructura
de pila.

El uso de la estructura de pila se debe a que los nodos que va encontrando
al último deben tener prioridad sobre los otros si es que se quiere explorar
en profundidad.

\paragraph{Búsqueda en amplitud}
Es un algoritmo que recorre los nodos de un grafo de manera más jerárquica,
esto es, que explora secuencialmente todos los nodos que están a una misma
distancia del nodo inicial, después de pasar a distancias más grandes. El
algoritmo encuentra la ruta más corta debido a esto.

Para lograr esto, es usual programar una cola, ya que los nodos vecinos
a un nodo se encuentran primero, por lo que deben tener mayor prioridad
que los vecinos de los nodos vecinos.

\section{Código}

\paragraph{Página}
\lstinputlisting[language=HTML]{../main.html}

\paragraph{Principal src/main.js}
\lstinputlisting[language=JavaScript]{../src/main.js}

\paragraph{Algoritmo src/word-morpher.js}
\lstinputlisting[language=JavaScript]{../src/word-morpher.js}

\paragraph{El que imprime la gráfica src/path.js}
\lstinputlisting[language=JavaScript]{../src/path.js}

\paragraph{El que hace las comparaciones src/dictionary.js}
\lstinputlisting[language=JavaScript]{../src/dictionary.js}

\section{Resultado}

\paragraph{Búsqueda en profundidad}
Con las palabras jump y drum ejecutadas 30 veces se obtuvo
una $\mu = 2.624s$ y $\sigma^2 = 0.159s$

\paragraph{Búsqueda en amplitud}
Con las palabras jump y drum ejecutadas 30 veces se obtuvo
una $\mu = 7.049s$ y $\sigma^2 = 1.478s$

\end{document}


